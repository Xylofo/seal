\documentclass[a4paper,titlepage,draft]{article}

% Swedish support
\usepackage[utf8]{inputenc}
\usepackage[T1]{fontenc}
\usepackage[english,swedish]{babel}

% Useful utilities
\usepackage{amsmath}
\usepackage{amsfonts}
\usepackage{amsthm}
\usepackage{amssymb}
\usepackage{graphicx}
\usepackage{microtype}
\usepackage{hyperref}
\usepackage{cleveref}
\usepackage{siunitx}
\usepackage{tikz}
\usepackage{pgfplots}
\usepackage{mathtools}
\usepackage{natbib}

\bibpunct{[}{]}{,}{n}{}{;}

\pgfplotsset{compat=1.10}

%====================  defined macros  ======================================

% \newcommand{\C}[1]{\Vert #1 \Vert}
\newcommand{\N}{\mathbb{N}}
\newcommand{\C}[1]{\mathfrak C \left( #1 \right)}

%====================  style theorems  ======================================

\newtheorem{theorem}{Sats}
\newtheorem{definition}{Definition}
\newtheorem{lemma}{Lemma}
\newtheorem{conjecture}{Förmodan}
\newtheorem{proposition}{Proposition}
\newtheorem{corollary}{Korollarium}
\newtheorem{statement}{Påstående}
\newtheorem{problem}{Problem}
\newtheorem{fact}{Faktum}

%====================  Document starts here  ======================================

\title{En studie av heltalens komplexitet}
\author{Lisa Vällfors \and Adrian Becedas \and Joakim Blikstad}

\begin{document}

\maketitle

\selectlanguage{english}
\begin{abstract}
    In this report we will study the complexity of N.
\end{abstract}
\selectlanguage{swedish}

\tableofcontents 
\newpage

\section{Inledning}

    \begin{definition}
       Vi benämner komplexiteten av ett positivt heltal $n$ som det minimala
       antalet $1$:or som krävs för att skriva $n$ med hjälp av addition,
       subtraktion och parenteser. Vi låter även $\C{n}$ beteckna komplexiteten av
       $n$. 
    \end{definition}

    \subsection{Bakgrund}

    \subsection{Frågeställningar}


\section{Analys}

    \subsection{Övre gräns}

    I detta avsnitt presenterar vi den övre gränsen upptäckt av J. Arias de
    Reyna~\citep{spansk}.
    \begin{definition}
        Vi definierar funktionen $A:\N\to\N$ på följande sätt:
        $$ A(n) = \left\{ \begin{matrix*}[l] 1 & \text{om } n=1 \\ 1+A(n-1) & \text{om $n$ är primtal} \\ \sum_{i=1}^kA(p_i) & \text{om } n=p_1p_2 \ldots p_k \end{matrix*} \right. $$
    \end{definition}

    Man ser att $A(n)\ge\C{n}$ då man alltid kan konstruera tal med
    funktionen~$A$. Värt att nämna är att det existerar $n$ där $A(n)\neq\C{n}$.
    Vi bevisar en övre gräns på $A(n)$ i syftet att få en övre gräns på $\C{n}$.

    \begin{lemma}
        $A(n)\le 3 \log_2{n}$ \quad för alla $n\ge2$
        \label{lemma:adrian}
    \end{lemma}
    \begin{proof}
        Vi utför stark induktion på $n$. Basfallet $n=2$ ger:
        \begin{align*}
            VL &=A(2)=1+A(1)=2\\
            HL &= 3 \log_2{2}=3 
        \end{align*}
        Alltså $VL \le HL$ som vi ville.
        Nu antar vi att \cref{lemma:adrian} gäller för alla $n \le k$. Vi vill
        visa att den även gäller för $n=k+1$.
        Vi har två fall, då $k+1$ är primtal och då det är sammansatt.

        $k+1$ är primtal ger enligt induktionsantagandet (notera $k$ är jämnt
        då $k+1$ är ett primtal $\ge3$):
        $$A(k+1) = 1 + A(k) = 1+2+A\left(\frac{k}{2}\right) \le 3 + 3 \log_2\frac{k}{2}$$
        Genom att använda identiteten \,$\log \frac{a}{b}=\log a -\log b$\, fås:
        $$ 3 + 3 \log_2\frac{k}{2} = 3 + 3\log_2 k - 3\log_2 2 = 3\log_2 k < 3\log_2 (k+1)$$
        Vilket ger att $A(k+1)\le 3\log_2 (k+1)$ om $k+1$ är ett primtal.

        Om $k+1$ är sammansatt kan vi skriva $k+1 = ab$ för heltal $a,b \ge 2$.
        Vi får då enligt induktionsantagandet:
        $$A(k+1) = A(a)+A(b) \le 3(\log_2a + \log_2b) = 3\log_2 ab = 3\log_2
        (k+1)$$ Alltså gäller även $A(k+1) \le 3\log_2 (k+1)$ om $k+1$ är
        sammansatt. Enligt induktionsprincipen är alltså \cref{lemma:adrian} nu
        bevisat.
    \end{proof}

    Beviset ger alltså att $\C{n}\le A(n)\le 3\log_2 n$ för alla $n\ge2$.




\section{Algoritmer för att generera sekvensen}

http://arxiv.org/pdf/1404.2183v2.pdf

För att generera data att undersöka, söks en algoritm för att beräkna komplexiteten av ett tal.

\subsection{En enkel algoritm}

	En första 

\bibliography{bib}{}
\bibliographystyle{unsrtnat}

\end{document}

