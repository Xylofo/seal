\documentclass[a4paperi,titlepage,onesided,draft]{article}

% Swedish support
\usepackage[utf8]{inputenc}
\usepackage[T1]{fontenc}
\usepackage[english,swedish]{babel}

% Useful utilities
\usepackage{graphicx}
\usepackage{microtype}
\usepackage{cleveref}
\usepackage{siunitx}
\usepackage{tikz}
\usepackage{pgfplots}

\pgfplotsset{compat=1.10}

\title{Rapport}
\author{Lisa Vällfors \and Adrian Becedas \and Joakim Blikstad}

\begin{document}

\maketitle

\selectlanguage{english}
\begin{abstract}
    In this report we will study the complexity of N.
\end{abstract}
\selectlanguage{swedish}

\tableofcontents 
\newpage

\section{Inledning}

\subsection{Bakgrund}

\subsection{Frågeställningar}


\section{Analys}

\begin{figure}[h]
\centering
\resizebox{\columnwidth}{!}{
    \input{normal}
}
\caption{histogram över de första 100\,000 värderna}
\label{fig:normal}
\end{figure}

Vi ser i \cref{fig:normal} att bla bla bla.

\section{Algoritmer för att generera sekvensen}

http://arxiv.org/pdf/1404.2183v2.pdf

För att generera data att undersöka, söks en algoritm för att beräkna komplexiteten av ett tal.

\subsection{En enkel algoritm}

En första 

\end{document}

