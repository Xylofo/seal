\documentclass[a4paperi,titlepage,onesided,draft]{article}

% Swedish support
\usepackage[utf8]{inputenc}
\usepackage[T1]{fontenc}
\usepackage[english,swedish]{babel}

% Useful utilities
\usepackage{amsmath}
\usepackage{amsfonts}
\usepackage{amsthm}
\usepackage{amssymb}
\usepackage{graphicx}
\usepackage{microtype}
\usepackage{cleveref}
\usepackage{siunitx}
\usepackage{tikz}
\usepackage{pgfplots}

\pgfplotsset{compat=1.10}

% \newcommand{\C}[1]{\Vert #1 \Vert}
\newcommand{\C}[1]{\frak C \left( #1 \right)}


%====================  style theorems  ======================================

\newtheorem{theorem}{Sats}
\newtheorem{definition}[theorem]{Definition}
\newtheorem{lemma}[theorem]{Lemma}
\newtheorem{conjecture}{Förmodan}
\newtheorem{proposition}[theorem]{Proposition}
\newtheorem{corollary}[theorem]{Korollarium}
\newtheorem{statement}[theorem]{Påstående}
\newtheorem{problem}{Problem}
\newtheorem{fact}[theorem]{Faktum}

%====================  Document starts here  ======================================

\title{En studie av heltalens komplexitet}
\author{Lisa Vällfors \and Adrian Becedas \and Joakim Blikstad}

\begin{document}

\maketitle

\selectlanguage{english}
\begin{abstract}
    In this report we will study the complexity of N.
\end{abstract}
\selectlanguage{swedish}

\tableofcontents 
\newpage

\section{Inledning}

\begin{definition}
   Vi benämner komplexiteten av ett positivt heltal $n$ som det minimala
   antalet $1$:or som krävs för att skriva $n$ med hjälp av addition,
   subtraktion och parenteser. Vi låter även $\C{n}$ beteckna komplexiteten av
   $n$. 
\end{definition}

\subsection{Bakgrund}

\subsection{Frågeställningar}


\section{Analys}

\section{Algoritmer för att generera sekvensen}

\end{document}

