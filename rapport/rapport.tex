\documentclass[a4paper,titlepage,draft]{article}

% Swedish support
\usepackage[utf8]{inputenc}
\usepackage[T1]{fontenc}
\usepackage[english,swedish]{babel}

% Useful utilities
\usepackage{amsmath}
\usepackage{amsfonts}
\usepackage{amsthm}
\usepackage{amssymb}
\usepackage{graphicx}
\usepackage{microtype}
\usepackage{hyperref}
\usepackage{cleveref}
\usepackage{siunitx}
\usepackage{tikz}
\usepackage{pgfplots}
\usepackage{mathtools}
\usepackage{natbib}
\usepackage{algpseudocode}

\bibpunct{[}{]}{,}{n}{}{;}

\pgfplotsset{compat=1.10}

%====================  defined macros  ======================================

% \newcommand{\C}[1]{\Vert #1 \Vert}
\newcommand{\N}{\mathbb{N}}
\newcommand{\C}[1]{\mathfrak C \left( #1 \right)}

%====================  style theorems  ======================================

\newtheorem{theorem}{Sats}
\newtheorem{definition}{Definition}
\newtheorem{lemma}{Lemma}
\newtheorem{conjecture}{Förmodan}
\newtheorem{proposition}{Proposition}
\newtheorem{corollary}{Korollarium}
\newtheorem{statement}{Påstående}
\newtheorem{problem}{Problem}
\newtheorem{fact}{Faktum}

%====================  Document starts here  ======================================

\title{En studie av heltalens komplexitet}
\author{Lisa Vällfors \and Adrian Becedas \and Joakim Blikstad}

\begin{document}

\maketitle

\selectlanguage{english}
\begin{abstract}
    In this report we will study the complexity of N.
\end{abstract}
\selectlanguage{swedish}

\tableofcontents 
\newpage

\section{Inledning}

    \begin{definition}
       Vi benämner komplexiteten av ett positivt heltal $n$ som det minimala
       antalet $1$:or som krävs för att skriva $n$ med hjälp av addition,
       subtraktion och parenteser. Vi låter även $\C{n}$ beteckna komplexiteten av
       $n$. 
    \end{definition}

    \subsection{Bakgrund}

    \subsection{Frågeställningar}


\section{Analys}

    \subsection{Övre gräns}

    I detta avsnitt presenterar vi den övre gränsen upptäckt av J. Arias de
    Reyna~\citep{spansk}.
    \begin{definition}
        Vi definierar funktionen $A:\N\to\N$ på följande sätt:
        $$ A(n) = \left\{ \begin{matrix*}[l] 1 & \text{om } n=1 \\ 1+A(n-1) & \text{om $n$ är primtal} \\ \sum_{i=1}^kA(p_i) & \text{om } n=p_1p_2 \ldots p_k \end{matrix*} \right. $$
    \end{definition}

    Man ser att $A(n)\ge\C{n}$ då man alltid kan konstruera tal med
    funktionen~$A$. Värt att nämna är att det existerar $n$ där $A(n)\neq\C{n}$.
    Vi bevisar en övre gräns på $A(n)$ i syftet att få en övre gräns på $\C{n}$.

    \begin{lemma}
        $A(n)\le 3 \log_2{n}$ \quad för alla $n\ge2$
        \label{lemma:adrian}
    \end{lemma}
    \begin{proof}
        Vi utför stark induktion på $n$. Basfallet $n=2$ ger:
        \begin{align*}
            VL &=A(2)=1+A(1)=2\\
            HL &= 3 \log_2{2}=3 
        \end{align*}
        Alltså $VL \le HL$ som vi ville.
        Nu antar vi att \cref{lemma:adrian} gäller för alla $n \le k$. Vi vill
        visa att den även gäller för $n=k+1$.
        Vi har två fall, då $k+1$ är primtal och då det är sammansatt.

        $k+1$ är primtal ger enligt induktionsantagandet (notera $k$ är jämnt
        då $k+1$ är ett primtal $\ge3$):
        $$A(k+1) = 1 + A(k) = 1+2+A\left(\frac{k}{2}\right) \le 3 + 3 \log_2\frac{k}{2}$$
        Genom att använda identiteten \,$\log \frac{a}{b}=\log a -\log b$\, fås:
        $$ 3 + 3 \log_2\frac{k}{2} = 3 + 3\log_2 k - 3\log_2 2 = 3\log_2 k < 3\log_2 (k+1)$$
        Vilket ger att $A(k+1)\le 3\log_2 (k+1)$ om $k+1$ är ett primtal.

        Om $k+1$ är sammansatt kan vi skriva $k+1 = ab$ för heltal $a,b \ge 2$.
        Vi får då enligt induktionsantagandet:
        $$A(k+1) = A(a)+A(b) \le 3(\log_2a + \log_2b) = 3\log_2 ab = 3\log_2
        (k+1)$$ Alltså gäller även $A(k+1) \le 3\log_2 (k+1)$ om $k+1$ är
        sammansatt. Enligt induktionsprincipen är alltså \cref{lemma:adrian} nu
        bevisat.
    \end{proof}

    Beviset ger alltså att $\C{n}\le A(n)\le 3\log_2 n$ för alla $n\ge2$.




\section{Algoritmer för att generera sekvensen}

http://arxiv.org/pdf/1404.2183v2.pdf

För att generera data att undersöka, söks en algoritm för att beräkna komplexiteten av ett tal.

\subsection{En enkel algoritm}

För att lösa frågan börjar vi baklänges, och ställer oss frågan: vilken ska den sista operationen vara?
Det är ju givet att representationen antingen kommer kunna sammanfattas som $n = a\cdot b$ eller $n = a+b$.
Vi vill därför för både
blabla rot, halv

\begin{algorithmic}[1]
\Procedure{complexity}{$n$}\Comment{Returnerar komplexiteten av n}
\If{$n = 1$}
\State \textbf{return} $1$
\EndIf
\State  $ans\gets \infty$
\For{$i\gets 1, n/2$}
\State $ans\gets \min(ans, \textsc{complexity}(i)+\textsc{complexity}(n-i))$
\EndFor
\For{$i\gets 2, \sqrt{n}$}
\If{$n\mod i = 0$}
\State $ans\gets \min(ans, \textsc{complexity}(i)+\textsc{complexity}(n/i))$
\EndIf
\EndFor

\EndProcedure
\end{algorithmic}

Denna algoritm finner utan problem komplexiteten av tal upp till runt 20. Därefter blir den alltför långsam.

En insikt kan dock förbättra vår metod och tillåta beräkning av komplexiteten för tal upp till 10 000 på under en sekund.

Problemet innehåller nämligen många överlappande delproblem, och dessa löser algoritmen ett flertal gånger. Om vi istället sparar resultatet för varje
tal kan de återanvändas. Vi ändrar nu även vår metod från att vara rekursiv, till att vara iterativ.


\begin{algorithmic}[1]
\Procedure{complexity}{$n$}\Comment{Returnerar komplexiteten av n}
\State Deklarera $v$
\For{$i\gets 1, n$}
\State \textbf{return} $1$
\State  $ans\gets \infty$
\For{$j\gets 1, n/2$}
\State $ans\gets \min(ans, v[i]+ v[n-i])$
\EndFor
\For{$i\gets 2, \sqrt{n}$}
\If{$n\mod i = 0$}
\State $ans\gets \min(ans, v[i]+v[n/i])$
\EndIf
\EndFor
\State $v[i] \gets ans$
\EndFor
\EndProcedure
\end{algorithmic}


Att algoritmen nu är ickerekursiv gör ingen skillnad i tidskomplexitet, men gör den i praktiken snabbare samt gör den något enklare att analysera.

Denna procedur har en loop med två loopar inuti. Detta gör att en beräkning av C(n) tar $n\left(\frac{n}{2}+\sqrt n\right) = \frac{n^2}{2} + n\sqrt n$. Eftersom vi räknar asymptotiskt spelar varken långsammare växande termer eller konstantfaktorer roll,  vilket innebär att algoritmen får tidskomplexiteten $\mathcal{O}(n^2)$

\bibliography{bib}{}
\bibliographystyle{unsrtnat}

\end{document}

